\chapter{Introduction}

Longitudinal studies are datasets that track information about multiple individuals over time.

Such studies are ``longitudinal'', in the sense that each individual has a sequence of data points that form a \emph{trajectory} through time. This makes longitudinal studies extremely valuable in examining patterns of change over time. Figure gives some examples of longitudinal data.


Compared to other types of data such as time series or regression data, longitudinal data have 
\begin{description}
\item[Different response types] It is quite common to collect responses of different types (e.g. continuous, discrete).
\item[Missing data] For large longitudinal studies, it is unlikely that every 
\end{description}

The analysis of longitudinal data can be divided into two parts: modelling and inference. In modelling, we construct a \emph{statistical model} that faithfully captures or explains the underlying patterns and processes in the data. In inference, we interpret our model in order to make predictions or conclusions. we use 
example on cows. A simple model would be to first find
- average
- linear regression
- In this thesis, we focus purely on the statistical modelling of longitudinal data. Inference tends to require domain-specific knowledge etc., so it is beyond the scope of this thesis.

There are many existing models for longitudinal data. We focus on a class of models called \emph{probabilistic models}, where we the probability distribution of a trajectory. Probabilistic models are useful because:
\begin{itemize}
\item You can answer queries
\item If you have several competing models, you can use Bayesian model selection.
\end{itemize}

Unfortunately, most existing probabilistic methods for longitudinal data make \emph{parametric assumptions}. In other words, they make assumptions about the distribution blah. For instance, the regression model in our cow milk example makes an assumption that the noise is Gaussian.

- Example why Gaussian assumptions could break.

In this thesis, we explore a class of probabilistic methods called \emph{non-parametric} models. There are many definitions of non-parametric, but here, we mean you do not have to specify a specific probability distribution a-priori in these models. In other words, the underlying distribution of the process is learned from training data on-the-fly.

The thesis is structured as follows: - background, -literature review -markov models, - noisy models, - latent models